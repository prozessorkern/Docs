\section{Vorwort}

Diese Kurze Anleitung soll den Einstieg in die Elektronik vereinfachen.
Sie stellt keinerlei Ansprüche auf Fehlerfreiheit und Vollständigkeit.
Ziel ist es grundlegende Maßnahmen und Techniken zu beschreiben, die beim Auffinden von Fehlern nützlich sein können.
Darüber hinaus soll ein Ansporn geschaffen werden durch eigenständige Weiterbildung aus öffentlich zugänglichen Quellen Fachwissen zur Lösung von Problemen aufzubauen (ingenieurmäßiges Arbeiten).


\section{Angesprochene Leser}

Diese Anleitung richtet sich vorwiegend an Mitarbeiter der IT Branche zum Aufbau von Knowhow zur Behebung von Problemen mit defekter PC Hardware z.B. zur Datenrettung von defekten Datenträgern.
Die beschriebenen Techniken sind jedoch größtenteils allgemein gehalten und lassen sich auf viele defekte Elektronik anwenden.

\section{Aufbau der Anleitung}

Dieses Dokument teilt sich folgendermaßen auf:
\begin{itemize}
	\item Werkzeug Grundausstattung
	\item Arbeitsschutz (sicheres Umfeld für sich und das zu Reparierende)
	\item Methoden zur Fehlersuche
	\item Häufige Fehlerquellen
	\item Reparatur Tipps
	\item Löten
\end{itemize}