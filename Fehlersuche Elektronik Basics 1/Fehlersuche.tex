\section{Fehlersuche}

\subsection{Ursachenforschung}

In der Elektronik ist häufig nicht leicht feststellbar, wo in einer komplexen Schaltung ein Fehler versteckt ist.
In vielen Fällen ist es daher hilfreich herauszufinden, was den Fehler ausgelöst haben könnte.
Soll heißen, durch welche äußeren Umstände könnte der Fehler erzeugt worden sein.

Im folgenden ein paar klassische Fehlerursachen und Probleme, die dadurch auftreten können:

\begin{itemize}
	\item Wasserschaden ==> Korrosion, Kurzschluss(Zerstörung von Bauteilen)
	\item Übertemperatur ==> Zerstörung von Bauteilen mit hoher Verlustleistung, Kontaktprobleme durch Verzug
	\item Mechanische Einwirkung ==> Bruch von Leiterbahnen, Brüche in Bauteilen, Kontaktprobleme
	\item Überspannung (Blitzeinschlag, Netzteildefekt) ==> Zerstörung von Schutzeinrichtungen, oder anderen Bauteilen
\end{itemize}



\subsection{optische Fehlersuche}

Die erste Analyse eines Fehlers sollte grundsätzlich durch eine genaue optische Fehlersuche erfolgen.
Auch wenn viele elektronische Fehler optisch nicht zu erkennen sind, führen einige Probleme doch zu sichtbaren Phänomenen.

Da elektrische Bauteile immer kleiner werden empfiehlt sich hier die Verwendung eines, wenn auch einfachen Mikroskops. Günstige Geräte mit USB Anschluss sind für schmales Geld erhältlich und erfüllen hier sehr gut Ihren Zweck.

Das genaue Augenmerk der optischen Fehlersuche hängt an dieser Stelle bereits von der vermuteten Ursache des Fehlers ab.

Auffälligkeiten:
\begin{itemize}
	\item Korrosion (sämtliche Bauteilanschlüsse absuchen, sofern zugänglich)
	\item Rückstände von Flüssigkeiten
	\item Brüche, Rissen der Leiterplatte, oder in Bauteilen
	\item Verfärbung von Teilen der Leiterplatte, oder von bestimmten Bauteilen
	\item sichtbare Beschädigungen an Bauteilen (häufig sichtbar durch kleine Löcher im Chipgehäuse) grundsätzlich sollte man sich jedes Bauteil, das in irgend einer Form merkwürdig aussieht näher untersuchen.
\end{itemize}

\subsection{messtechnische Fehlersuche}

Spannungen verfolgen

Kurzschlüsse messen

Dioden messen

Kapazitäten nachmessen (funktioniert selten gut in eingebautem Zustand)

\subsection{häufige Fehlerquellen}

Elkos

Schutzeinrichtungen(Sicherungen, Schutzdioden)

Mechanische/Kontaktprobleme (abgerissene Steckverbinder, Korrosion)

\subsection{Achtung Folgefehler}

Wenn man ein Problem löst, Vorsicht, evtl. liegt das eigentliche Problem erst dahinter...